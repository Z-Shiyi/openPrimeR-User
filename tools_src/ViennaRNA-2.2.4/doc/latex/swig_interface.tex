\hypertarget{swig_interface_swig_intro}{}\section{Introduction}\label{swig_interface_swig_intro}
For an easy integration into scripting languages, we provide an automatically generated interface to the R\+N\+Alib C-\/library, generated with swig.\hypertarget{swig_interface_swig_renaming}{}\subsection{Function renaming scheme}\label{swig_interface_swig_renaming}
The main difference when using a scripting language interface compared to direct calls of R\+N\+Alib C functions is, that the prefix 'vrna\+\_\+' is dropped. For instance, when calling the \hyperlink{group__mfe__fold__single_gae7ca49ffb3086f145da36c964a7cec64}{vrna\+\_\+fold()} function, corresponding calls in perl or python are R\+N\+A\+::fold(), and R\+N\+A.\+fold(), respectively.

Functions that are dedicated to work on specific data structures only, e.\+g. the \hyperlink{group__fold__compound_ga1b0cef17fd40466cef5968eaeeff6166}{vrna\+\_\+fold\+\_\+compound\+\_\+t}, are usually not exported at all. Instead, they are attached as object methods of a corresponding class (see \hyperlink{swig_interface_swig_oo_interface}{Object oriented Interface for data structures} for detailed information).\hypertarget{swig_interface_swig_oo_interface}{}\subsection{Object oriented Interface for data structures}\label{swig_interface_swig_oo_interface}
For data structures, typedefs, and enumerations the 'vrna\+\_\+' prefixes are dropped as well, together with their suffixes '\+\_\+s', '\+\_\+t', and '\+\_\+e', respectively. Furthermore, data structures are usually transformed into classes and relevant functions of the C-\/library are attached as methods.\hypertarget{swig_interface_swig_examples}{}\section{Examples}\label{swig_interface_swig_examples}
\hypertarget{swig_interface_swig_perl_examples}{}\subsection{Perl Examples}\label{swig_interface_swig_perl_examples}
\hypertarget{swig_interface_swig_perl_examples_flat}{}\subsubsection{Using the Flat Interface}\label{swig_interface_swig_perl_examples_flat}
Example 1\+: \char`\"{}\+Simple M\+F\+E prediction\char`\"{} 
\begin{DoxyCodeInclude}
00001 #!/usr/bin/perl
00002 
00003 use warnings;
00004 use strict;
00005 
00006 use RNA;
00007 
00008 my $seq1 = "CGCAGGGAUACCCGCG";
00009 
00010 # compute minimum free energy (mfe) and corresponding structure
00011 my ($ss, $mfe) = RNA::fold($seq1);
00012 
00013 # print output
00014 printf "%s [ %6.2f ]\(\backslash\)n", $ss, $mfe;
\end{DoxyCodeInclude}
\hypertarget{swig_interface_swig_perl_examples_oo}{}\subsubsection{Using the Object Oriented (\+O\+O) Interface}\label{swig_interface_swig_perl_examples_oo}
The 'fold\+\_\+compound' class that serves as an object oriented interface for \hyperlink{group__fold__compound_ga1b0cef17fd40466cef5968eaeeff6166}{vrna\+\_\+fold\+\_\+compound\+\_\+t}

Example 1\+: \char`\"{}\+Simple M\+F\+E prediction\char`\"{} 
\begin{DoxyCodeInclude}
00001 #!/usr/bin/perl
00002 
00003 use warnings;
00004 use strict;
00005 
00006 use RNA;
00007 
00008 my $seq1 = "CGCAGGGAUACCCGCG";
00009 
00010 # create new fold\_compound object
00011 my $fc = new RNA::fold\_compound($seq1);
00012 
00013 # compute minimum free energy (mfe) and corresponding structure
00014 my ($ss, $mfe) = $fc->mfe();
00015 
00016 # print output
00017 printf "%s [ %6.2f ]\(\backslash\)n", $ss, $mfe;
\end{DoxyCodeInclude}
\hypertarget{swig_interface_swig_python_examples}{}\subsection{Python Examples}\label{swig_interface_swig_python_examples}
